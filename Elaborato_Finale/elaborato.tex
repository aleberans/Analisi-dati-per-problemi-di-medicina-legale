\documentclass[11pt,a4paper]{book}
\usepackage[utf8]{inputenc}
\usepackage[english]{babel}
\usepackage{amsmath}
\usepackage{amsfonts}
\usepackage{amssymb}
\usepackage[left=3.50cm, right=3.00cm, top=3.00cm, bottom=3.00cm]{geometry}
\author{Alessandro Beranti}
\title{Analisi dei Dati per Problemi di Medicina Legale}
\begin{document}
	\maketitle
	\chapter{Introduzione}
	\chapter{Machine Learning}
		\section{Come funziona il Machine Learning}
			\subsection{Machine Learning con Apprendimento Supervisionato}
			\subsubsection{Support Vector Machine}
			\subsubsection{Decision Tree Classiier}
			\subsubsection{Random Forest Classifier}
			\subsubsection{Gaussian Naive Bayes}
			\subsubsection{Linear Discriminat Analysis}
			\subsubsection{Multi-Layer Perceptron Classifier}
		\section{Machine Learning con apprendimento Non Supervisionato}
		\section{Machine Learning con apprendimento per Rinforzo}
		\section{Machine Learning con apprendimento semi Supervisionato}
	\chapter{Dataset}
		\section{Iris}
		\section{Incidenti Stradali}
		\section{Metodi per ridurre la Dimensionalità}
		
	\chapter{Esperimenti}
	\chapter{Conclusioni}
\end{document}