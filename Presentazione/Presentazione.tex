\documentclass{beamer}
%
% Choose how your presentation looks.
%
% For more themes, color themes and font themes, see:
% http://deic.uab.es/~iblanes/beamer_gallery/index_by_theme.html
%
\mode<presentation>
{
  \usetheme{default}      % or try Darmstadt, Madrid, Warsaw, ...
  \usecolortheme{default} % or try albatross, beaver, crane, ...
  \usefonttheme{default}  % or try serif, structurebold, ...
  \setbeamertemplate{navigation symbols}{}
  \setbeamertemplate{caption}[numbered]
} 

\usepackage[italian]{babel}
\usepackage[utf8]{inputenc}
\usepackage[T1]{fontenc}
\usepackage{booktabs}
\usepackage{adjustbox}
\usepackage{geometry}
\usepackage{xcolor,colortbl}
\usepackage{array, booktabs, makecell}
\usepackage{siunitx, mhchem}

\title[Your Short Title]{Incidenti Stradali}
\author{Alessandro Beranti}
\institute{Università degli Studi di Milano Statale }
\date{25 Febbraio 2020}

\begin{document}

\begin{frame}
  \titlepage
\end{frame}

% Uncomment these lines for an automatically generated outline.
%\begin{frame}{Outline}
%  \tableofcontents
%\end{frame}

\section{Introduction}

\begin{frame}{Introduzione}

\begin{block}{Obiettivo}
	Stabilire che tipo di mezzo ha investito la vittima
\end{block}

\begin{block}{Caratteristiche iniziali}
A partire dal file fornito ho eseguito diversi studi considerando di volta in volta diversi fattori:
\end{block}

\begin{itemize}
	\item Caratteristiche delle ossa rotte
	\item Algoritmi di apprendimento
	\item Pre-processing dei dati tramite: 
	\begin{itemize}
		\item StandardScaler
		\item MinMaxScaler
		\item RobustScaler
	\end{itemize}
\end{itemize}


\begin{block}{Risultati ottenuti}
Da ogni studio sono conseguite diverse tabelle che tengono conto dei fattori precedentemente descritti.
\end{block}

\end{frame}

\begin{frame}{Modelli Apprendimento}
\begin{block}{Modelli usati}
	Per compiere lo studio sono stati usati diversi modelli di apprendimento:
	\begin{itemize}
		\item Support Vector Machine ( SVC )
		\item Decision Tree ( DT )
		\item Random Forest ( RF )
		\item GaussianNB (NB )
		\item LinearDiscriminantAnalysis ( LD )
		\item MLPClassifier ( MLP )
	\end{itemize}
\end{block}

\end{frame}

\begin{frame}{Componenti usate}
	\begin{block}{}
		Totali:
		\begin{itemize}
			\item Sesso, Anni, Peso, Altezza, Tot Testa, Tot Torace, Tot Addome, Tot Scheletro
		\end{itemize}
	\end{block}
	\begin{block}{}
		Totali\_BMI:
		\begin{itemize}
			\item Sesso, Anni, Peso, Altezza, Bmi, Tot Testa, Tot Torace, Tot Addome, Tot Scheletro
		\end{itemize}
	\end{block}
	\begin{block}{}
		Totali\_DATA:
		\begin{itemize}
			\item Data, Sesso, Anni, Peso, Altezza, Tot Testa, Tot Torace, Tot Addome, Tot Scheletro
		\end{itemize}
	\end{block}
\end{frame}

\begin{frame}{Componenti usate}
	\begin{block}{}
	Totali\_BMI\_DATA:
		\begin{itemize}
		\item Data, Sesso, Anni, Peso, Altezza, Bmi, Tot Testa, Tot Torace, Tot Addome, Tot Scheletro
		\end{itemize}
	\end{block}
	\begin{block}{}
		Details, composto da:
		\begin{columns}[t]
			\begin{column}{0.5\textwidth}
				\begin{itemize}
					\item Testa
					\begin{itemize}
						\item Neurocranio
						\item Splancnocranio
						\item Telencefalo
						\item Cervelletto
						\item Tronco-encefalico
					\end{itemize}
				\end{itemize}
			\end{column}
			\begin{column}{0.5\textwidth}
				\begin{itemize}
					\item Torace
					\begin{itemize}
						\item Polmoni
						\item Trachea/bronchi
						\item Cuore
						\item Aorta-toracica
						\item Diaframma
					\end{itemize}
				\end{itemize}
			\end{column}
		\end{columns}
	\end{block}
\end{frame}

\begin{frame}{Componenti usate}
	\begin{columns}
		\begin{column}{0.5\textwidth}
			\begin{itemize}
				\item Addome
				\begin{itemize}
					\item Fegato
					\item Milza
					\item Aorta-addominale
					\item Reni
					\item Mesentere
				\end{itemize}
			\end{itemize}
		\end{column}
		\begin{column}{0.5\textwidth}
			\begin{itemize}
				\item Scheletro
				\begin{itemize}
					\item Rachide-cervicale
					\item Rachide-toracico
					\item Rachide-lombare
					\item bacino-e-sacro
					\item Complesso-sterno/claveo/costale
				\end{itemize}
			\end{itemize}
		\end{column}
	\end{columns}
\end{frame}

\begin{frame}{Analizziamo i risultati}
	\begin{itemize}
		\item PCA e TSNE sono gli algoritmi usati per ridurre la dimensionalità
		\item Il numero a fianco indica a quante dimensioni si è scesi
		\item Details ha una dimensione iniziale di 20
		\item Il punteggio sta ad indicare la percentuale di casi in cui l'algoritmo ha predetto correttamente di che mezzo si tratta
	\end{itemize}

\end{frame}

\begin{frame}{Analizziamo i risultati}
\begin{block}{StandardScaler}
	\begin{center}
		\begin{adjustbox}{max width=\textwidth}
			\begin{tabular}{lrrrr}
				\toprule
				{} &    Totali &  Totali\_BMI &  Totali\_DATA &  Totali\_DATA\_and\_BMI \\
				\midrule
				SVC &  0.66 &         0.66 &          0.77 &                  0.73 \\
				DT  &  0.56 &         0.56 &          0.61 &                  0.63 \\
				RF  &  0.64 &         0.63 &          0.57 &                  0.59 \\
				NB  &  0.69 &         0.68 &          0.72 &                  0.75 \\
				LD  &  0.71 &         0.68 &          0.72 &                  0.72 \\
				MLP &  0.69 &         0.69 &          0.74 &                  0.72 \\
				\bottomrule
			\end{tabular}
		\end{adjustbox}
	\end{center}
	\begin{center}
		\begin{adjustbox}{max width=\textwidth}
			\begin{tabular}{lrrrrrrr}
				\toprule
				{} &   Details &  \thead{Details\\PCA\_5} & \thead{Details\\ PCA\_10} &  \thead{Details\\ PCA\_13} &  \thead{Details\\PCA\_15} &  \thead{Details\\TSNE\_13} &  \thead{Details\\TSNE\_15} \\
				\midrule
				SVC &  0.68 &                  0.63 &                   0.68 &                   0.67 &                   0.71 &                    0.51 &                    0.53 \\
				DT  &  0.68 &                  0.60 &                   0.67 &                   0.64 &                   0.57 &                    0.43 &                    0.50 \\
				RF  &  0.67 &                  0.59 &                   0.69 &                   0.70 &                   0.63 &                    0.51 &                    0.55 \\
				NB  &  0.63 &                  0.61 &                   0.66 &                   0.67 &                   0.65 &                    0.56 &                    0.57 \\
				LD  &  0.63 &                  0.64 &                   0.60 &                   0.59 &                   0.59 &                    0.48 &                    0.49 \\
				MLP &  0.63 &                  0.62 &                   0.61 &                   0.62 &                   0.65 &                    0.39 &                    0.47 \\
				\bottomrule
			\end{tabular}
		\end{adjustbox}
	\end{center}
\end{block}
\end{frame}
\begin{frame}{Analizziamo i risultati}
	\begin{block}{MinMaxScaler}
		\begin{center}
			\begin{adjustbox}{max width=\textwidth}
				\begin{tabular}{lrrrr}
					\toprule
					{} &    Totali &  Totali\_BMI &  Totali\_DATA &  Totali\_DATA\_and\_BMI \\
					\midrule
					SVC &  0.66 &         0.62 &          0.73 &                  0.74 \\
					DT  &  0.61 &         0.55 &          0.65 &                  0.63 \\
					RF  &  0.60 &         0.62 &          0.62 &                  0.61 \\
					NB  &  0.69 &         0.68 &          0.72 &                  0.75 \\
					LD  &  0.71 &         0.68 &          0.72 &                  0.72 \\
					MLP &  0.62 &         0.64 &          0.71 &                  0.76 \\
					\bottomrule
				\end{tabular}
			\end{adjustbox}
		\end{center}
		\begin{center}
			\begin{adjustbox}{max width=\textwidth}
				\begin{tabular}{lrrrrrrr}
					\toprule
					{} &   Details &  \thead{Details\\PCA\_5} &  \thead{Details\\PCA\_10} &  \thead{Details\\PCA\_13} &  \thead{Details\\PCA\_15} &  \thead{Details\\TSNE\_13} &  \thead{Details\\TSNE\_15} \\
					\midrule
					SVC &  0.74 &                  0.70 &                   0.64 &                   0.67 &                   0.70 &                    0.58 &                    0.42 \\
					DT  &  0.70 &                  0.61 &                   0.60 &                   0.56 &                   0.59 &                    0.51 &                    0.56 \\
					RF  &  0.67 &                  0.65 &                   0.66 &                   0.71 &                   0.67 &                    0.46 &                    0.52 \\
					NB  &  0.63 &                  0.67 &                   0.65 &                   0.67 &                   0.67 &                    0.57 &                    0.58 \\
					LD  &  0.63 &                  0.65 &                   0.61 &                   0.60 &                   0.62 &                    0.43 &                    0.39 \\
					MLP &  0.65 &                  0.64 &                   0.64 &                   0.60 &                   0.63 &                    0.51 &                    0.43 \\
					\bottomrule
				\end{tabular}
			\end{adjustbox}
		\end{center}
	\end{block}
\end{frame}
\begin{frame}{Analizziamo i risultati}
	\begin{block}{RobustScaler}
		\begin{center}
			\begin{adjustbox}{max width=\textwidth}
				\begin{tabular}{lrrrr}
					\toprule
					{} &    Totali &  Totali\_BMI &  Totali\_DATA &  Totali\_DATA\_and\_BMI \\
					\midrule
					SVC &  0.66 &         0.68 &          0.77 &                  0.72 \\
					DT  &  0.59 &         0.60 &          0.62 &                  0.65 \\
					RF  &  0.62 &         0.63 &          0.56 &                  0.61 \\
					NB  &  0.69 &         0.68 &          0.72 &                  0.75 \\
					LD  &  0.71 &         0.68 &          0.72 &                  0.72 \\
					MLP &  0.70 &         0.67 &          0.70 &                  0.66 \\
					\bottomrule
				\end{tabular}
			\end{adjustbox}
		\end{center}
		\begin{center}
			\begin{adjustbox}{max width=\textwidth}
				\begin{tabular}{lrrrrrrr}
					\toprule
					{} &   Details &  \thead{Details\\PCA\_5} &  \thead{Details\\PCA\_10} &  \thead{Details\\PCA\_13} &  \thead{Details\\PCA\_15} &  \thead{Details\\TSNE\_13} &  \thead{Details\\TSNE\_15} \\
					\midrule
					SVC &  0.68 &                  0.63 &                   0.63 &                   0.66 &                   0.66 &                    0.56 &                    0.47 \\
					DT  &  0.65 &                  0.63 &                   0.61 &                   0.64 &                   0.63 &                    0.45 &                    0.56 \\
					RF  &  0.65 &                  0.65 &                   0.64 &                   0.64 &                   0.66 &                    0.52 &                    0.49 \\
					NB  &  0.63 &                  0.67 &                   0.67 &                   0.67 &                   0.65 &                    0.54 &                    0.56 \\
					LD  &  0.63 &                  0.63 &                   0.60 &                   0.64 &                   0.67 &                    0.43 &                    0.42 \\
					MLP &  0.56 &                  0.63 &                   0.60 &                   0.65 &                   0.60 &                    0.49 &                    0.46 \\
					\bottomrule
				\end{tabular}
			\end{adjustbox}
		\end{center}
	\end{block}
\end{frame}
\begin{frame}{Analizziamo i risultati}
	\begin{block}{Senza Scaler}
		\begin{center}
			\begin{adjustbox}{max width=\textwidth}
				\begin{tabular}{lrrrrrr}
					\toprule
					{} &  \thead{Details\\PCA\_5} &  \thead{Details\\PCA\_10} &  \thead{Details\\PCA\_13} &  \thead{Details\\PCA\_15} &  \thead{Details\\TSNE\_13} &  \thead{Details\\TSNE\_15} \\
					\midrule
					SVC &                  0.65 &                   0.66 &                   0.68 &                   0.66 &                    0.57 &                    0.54 \\
					DT  &                  0.60 &                   0.55 &                   0.59 &                   0.51 &                    0.50 &                    0.52 \\
					RF  &                  0.66 &                   0.64 &                   0.60 &                   0.64 &                    0.53 &                    0.55 \\
					NB  &                  0.67 &                   0.64 &                   0.66 &                   0.65 &                    0.52 &                    0.55 \\
					LD  &                  0.64 &                   0.62 &                   0.58 &                   0.62 &                    0.47 &                    0.51 \\
					MLP &                  0.66 &                   0.65 &                   0.60 &                   0.58 &                    0.43 &                    0.53 \\
					\bottomrule
				\end{tabular}
			\end{adjustbox}
		\end{center}
	In generale non c’è un aumento di prestazioni tra i vari scaler
	\end{block}
\end{frame}
\begin{frame}{Conclusioni}
	\begin{block}{}
		\begin{center}
			\begin{adjustbox}{max width=\textwidth}
				\begin{tabular}{lrrrr}
					\toprule
					{} &  StandardScaler &  MinMaxScaler &  RobustScaler &  No\_Scaler \\
					\midrule
					Totali                     &        0.713 &      0.713 &      0.713 &        NaN \\
					Totali\_BMI            &        0.692 &      0.689 &      0.689 &        NaN \\
					Totali\_DATA           &        \cellcolor{blue!25}0.774 &      0.735 &      \cellcolor{blue!25}0.774 &        NaN \\
					Totali\_DATA\_BMI   &        0.751 &      0.767 &      0.751 &        NaN \\
					Details                    &        0.685 &      0.740 &      0.685 &        NaN \\
					Details\_PCA\_5   &        0.647 &      0.701 &      0.670 &   0.670 \\
					Details\_PCA\_10  &        0.699 &      0.662 &      0.670 &   0.662 \\
					Details\_PCA\_13  &        0.701 &      0.710 &      0.670 &   0.685 \\
					Details\_PCA\_15  &        0.715 &      0.700 &      0.678 &   0.661 \\
					Details\_TSNE\_13 &        0.561 &      0.584 &      0.568 &   0.572 \\
					Details\_TSNE\_15 &        0.578 &      0.586 &      0.562 &   0.554 \\
					\bottomrule
				\end{tabular}
			\end{adjustbox}
		\end{center}
	\end{block}
	Le migliori prestazioni si ottengono usando Totali con la data senza praticare una riduzione della dimensionalità
\end{frame}
\end{document}